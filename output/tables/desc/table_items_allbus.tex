% Please add the following required packages to your document preamble:
% \usepackage{booktabs}
% \usepackage{graphicx}
\begin{center}
\begin{small}
\begin{longtable}{p{8cm}p{4cm}}
\caption{Description of used items from the Survey of German citizens (ALLBUS)}
\label{tab:items_allbus}
\\
\hline
\hline
Description & Range \\ 
\hline
\endfirsthead
\multicolumn{2}{c}%
{\tablename\ \thetable\ -- \textit{Continued from previous page}} \\
\hline
\hline
Description & Range \\ 
\hline
\endhead
\hline \multicolumn{2}{r}{\textit{Continued on next page}} \\
\endfoot
\hline
\hline
\endlastfoot
\multicolumn{2}{l}{\textbf{Emotions}}  \vspace{0.2cm} \\
\multicolumn{2}{p{13cm}}{\textit{What about asylum seekers/Turkish people/Italian people/Jewish people/Polish people living in Germany? To what extent do the following statements apply:}}  \\
...I feel sorry for them. & 1 (Applies completely) to 4 (Does not apply at all).\\
...They annoy me. &  1 (Applies completely) to 4 (Does not apply at all).\\
...I find them likeable. &  1 (Applies completely) to 4 (Does not apply at all).\\
...They scare me. &  1 (Applies completely) to 4 (Does not apply at all).\\ 
\hline
\multicolumn{2}{l}{\textbf{Perceived risks}}  \vspace{0.2cm} \\
\multicolumn{2}{p{13cm}}{\textit{If you think about the development of German society in the next few years: Do you think that, in the following areas, there will be more opportunities, more risks or neither of these as a result of the refugees?}}  \\
...As regards the welfare state. & 1 (Considerably more risks) to 5 (Considerably more opportunities). \\
...As regards public security. &  1 (Considerably more risks) to 5 (Considerably more opportunities). \\
...As regards people living together in society. &  1 (Considerably more risks) to 5 (Considerably more opportunities). \\
...As regards the economic situation in Germany. &  1 (Considerably more risks) to 5 (Considerably more opportunities).\\ 
\hline
\multicolumn{2}{l}{\textbf{Social Distance}}  \vspace{0.2cm} \\
How pleasant or unpleasant would it be to for you to have an asylum seeker/Turkish person/Italian person/Jewish person/Polish person as your neighbour? & 1 (Very pleasant) to 7 (Very unpleasant). \\
How strongly, in your opinion, do asylum seekers/Turkish persons/Italian persons/Jewish persons/Polish persons who live in Germany differ from Germans in their lifestyles? &  1 (Not at all) to 7 (Very strongly). \\
\end{longtable}
\end{small}
\end{center}
